\documentclass[../main.tex]{subfiles}

\begin{document}
%% begin abstract format
\makeatletter
\renewenvironment{abstract}{%
    \if@twocolumn
      \section*{Abstract \\}%
    \else %% <- here I've removed \small
    \begin{flushright}
        {\filleft\Huge\bfseries\fontsize{48pt}{12}\selectfont Abstract\vspace{\z@}}%  %% <- here I've added the format
        \end{flushright}
      \quotation
    \fi}
    {\if@twocolumn\else\endquotation\fi}
\makeatother
%% end abstract format
\begin{abstract}

Most of the students who takes physics lectures at college related with engineering or architecture, often find some complexity to understand topics that they are studying. Math difficulty is not the main problem; but rather, their application on the physical phenomena and the interpretation of the their evolution.\\

In order to minimize this problem and taking the RC and RL circuits, commonly used in electrical and electronic devices as study systems, we propose the development of a web application capable of clearly showing the theoretical expressions that explain their behavior, analyzing the temporal evolution of the characterictics physical magnitudes of these systems when they work in direct current and transitory state. \\

We will approach this problem using latest web development technologies, where we will use programming languages such as \textit{JavaScript} and enviroments like \textit{NodeJS}. In addition, a \textit{python} script will be implemented to verify that results generated by the final  application are correct. This program will generate an image with the results that we should expect from the simulation.\\ 

Finally, we show the conclusions about the development and analysis of this application proposing its possible adaptation to other physical systems similar to the studied circuits.\\



\bfseries{\large{Keywords:}} simulation, RC circuit, RL circuit, direct current, transient state, ReactJS, web application

\end{abstract}
\end{document}