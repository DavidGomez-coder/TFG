\documentclass[../main.tex]{subfiles}

\begin{document}

\subsection{Conclusions}
On first chapter, we defined the main goal of this project: develop teaching materials for physics students who wants to reinforce understanding of RC and RL first order circuits, by the way of computer simulations.\\ 

On the development of these simulations, three different types of applications were considered: native, hybrid and web apps, where we selected web applications for their usability on different market devices. After that, we started with the development, fulfilling the following objectives on each stage.

\begin{itemize}
    \item On analysis stage, the dialogue with both teachers was quite important, since we extracted all the functional requirements (table 6) that would be integrated into the application. In addition, a  basic design idea was came up. \\
    
    At the same time, an investigation about the proposed physical phenomena was carried out, in which maths expressions were modeled.

    \item After analysis, we considered several candidates as possible technologies, such as application type or programming languajes. At the begining, the objective was Java as main programming language, but we decided to use latest web development technologies instead.\\ 
    
    The stage main goal to achieve was learn to master the selected \textit{frameworks} and programming languages that will be use at implementation.

    \item Once we have learned the necessary technologies, we come to the implementation and testing stage. On this stage we use technologies we have learned and mathematical expressions that we get during the circuits analysis.
    
    \item Once the development is complete, we perform an application deployment. The goal is to make it public, so anyone could access it from a web browser. This could be considered the easiest task just because programming enviroment we're using allows to integrate a deployment option with GitHub.
    


\end{itemize}

With all this, we have a fully functional web application, accesible from any device with a web browser. The intuitive user interface allow the user to navigate through it easily. In addition, along with the simulations, there're theories sections that reinforce the teaching objective, fully simulating a theoretical-practical class of RC and RL circuits in direct current and transient state.


\subsection{Future lines of Research}
As future lines, the following ones can be proposed.

\begin{itemize}
    \item \textbf{Physics main page}. One of the lines to be proposed would be the development of a web portal wehre all information about the subject will be collected; for example, simulations, theoretical and practical sections, a login system that allows to enter only subject students, etc.
    
    \item \textbf{RC and RL circuits simulation on direct and alternating current}. Main goal of this line would be to learn how RC and RL circuits works, where the app user could build its own circuit. It coulud be interesting to show how differents current intensity, capacitors charge or magnetic flux at inductor evolves on time. The \textbf{simulink API} could be used for this purpose.
    
    \item \textbf{Physics simulations for teaching}. This line of research has the objective to extend this one, adding new simulations to the existing ones, such as the interaction between electric charges.
\end{itemize}

\end{document}