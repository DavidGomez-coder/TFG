\documentclass[../main.tex]{subfiles}

\begin{document}

\subsection{Conclusions}

On first chapter, we defined the main goal of this project: develop teaching materials for physics students who wants to reinforce understanding of RC and RL first order circuits, by the way of computer simulations.


\begin{itemize}
    
    \item The following applications types have been analyzed: native, hybrid and web apps, where we selected web applications for their usability on different market devices. After that, we started with the development, fulfilling the following objectives on each stage.
    
    \item We develop one first stage based on academics needs analysis, where we extracted all the functional requirements (table \ref{tab::requisitos}) and the design that have been integrated into the application;  mainly, the possibility of visualizing in real time the evolution of the electrical magnitudes of each circuit.    

    \item After an exhaustive analysis of the application, a study of the different technologies to be used, such as the type of application and programming languages, was carried out. Originally Java were considered to be the main programming language, but finally, we decided to use current web development technologies.
    
    In this stage, we learned to master the differents framewoks and languages used in the implementation. Among these technologies, we find \textit{ReactJS}, a frontend framework oriented to views programming based on components. \textit{ReactJS} is written with \textit{JavaScrip}, alongside \textit{HTML} and \textit{CSS}.
        
    Also, we use web version of \textit{git}: \textit{GitHub}, to have a control of the differents versions of the project. GitHub also accept a configuration that allows code deployment, giving us access from any web browser.

    \item The implementation of the application has been carried out, where the most outstanding feature is the algorithm used in data generation, in which we take advantage of component's life cycle in react to generate, process and represent the results.
    
    \item It has been verified that the application is in accordance with the requirements. This is a completely visual application and what we are interested in is that the graphs generated are correct, so tests are a comparison between the results generated by the application and those obtained by a \textit{python script}.
    
    \item Once the development is complete, we perform an application deployment. The goal is to make it public, so anyone could access it from a web browser.
    
    \item With all this, we have a fully functional web application, accesible from any device with a web browser. The intuitive user interface allow the user to navigate through it easily. In addition, along with the simulations, there're theories sections that reinforce the teaching objective, fully simulating a theoretical-practical class of RC and RL circuits in direct current and transient state.

    


\end{itemize}



\subsection{Future lines of Research}
As future lines, the following ones can be proposed.

\begin{itemize}
    \item \textbf{Physics main page}. One of the lines to be proposed would be the development of a web portal wehre all information about the subject will be collected; for example, simulations, theoretical and practical sections, a login system that allows to enter only subject students, etc.
    
    \item \textbf{RC and RL circuits simulation on direct and alternating current}. Main goal of this line would be to learn how RC and RL circuits works, where the app user could build its own circuit. It could be interesting to show how differents current intensity, capacitors charge or magnetic flux at inductor evolves on time. The \textbf{simulink API} could be used for this purpose. At differential equations, we should consider the type of current we are using, so a new mathematical analysis for these circuits will be necessary. 
    
    \item \textbf{Physics simulations for teaching}. This line of research has the objective to extend this one, adding new simulations to the existing ones, such as the interaction between electric charges, or add new devices to RC and RL circuits like diodes or transistors.
\end{itemize}

\end{document}