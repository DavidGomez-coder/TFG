\documentclass[../main.tex]{subfiles}

\begin{document}

\subsection{Conclusions}

On first chapter, we defined the main goal of this project: develop teaching materials for physics students who wants to reinforce understanding of RC and RL first order circuits, by the way of computer simulations.\\ 




\begin{itemize}
    
    \item One of the first conclusion is the type of application that we need to select, evaluating one of the following three: native, hybrid and web apps, where we selected web applications for their usability on different market devices. After that, we started with the development, fulfilling the following objectives on each stage.
    
    \item We develop one first stage based on academics needs analysis, where we extracted all the functional requirements (table \ref{tab::requisitos}) and the design that have been integrated into the application;  mainly, the possibility of visualizing in real time the evolution of the electrical magnitudes of each circuit.\\
    
    The application allows to select the values of each circuit component within an interval of common use in the real laboratory, while we can visualize the mathematical expresions that justify the behavior of the circuits.

    \item After analysis, we considered several candidates for the technologies to use, such as application type or programming languages. At the begining, we evaluate Java as the main programming language, but we decided to use latest web development technologies instead.\\
    
    This stage main goal was the achievement to learn how the selected technologies works.\\ 

    Among these technologies, we find \textit{ReactJS}, a frontend framework oriented to views programming based on components. One feature of why this is commonly used is because of its learning difficulty, being easiest than others frameworks with the same purpose. \textit{ReactJS} is written with \textit{JavaScrip
    }, so it is necessary learn to master it, alongside \textit{HTML} and \textit{CSS}.

    In addition, we use web version of \textit{git}: \textit{GitHub}, to make control of the differents versions of the project. GitHub also accept a configuration that allows code deployment, giving us access from any web browser.
    
    \item All applications need to be tested to ensure that it runs agree with requirements. The web application we made is mostly visual, so all tests are about to compare the results generated by our application with those obtained by a \textit{python} script.
   
    \item Once the development is complete, we perform an application deployment. The goal is to make it public, so anyone could access it from a web browser. This could be considered the easiest task just because programming enviroment we are using allows to integrate a deployment option with GitHub.
    


\end{itemize}

With all this, we have a fully functional web application, accesible from any device with a web browser. The intuitive user interface allow the user to navigate through it easily. In addition, along with the simulations, there're theories sections that reinforce the teaching objective, fully simulating a theoretical-practical class of RC and RL circuits in direct current and transient state.


\subsection{Future lines of Research}
As future lines, the following ones can be proposed.

\begin{itemize}
    \item \textbf{Physics main page}. One of the lines to be proposed would be the development of a web portal wehre all information about the subject will be collected; for example, simulations, theoretical and practical sections, a login system that allows to enter only subject students, etc.
    
    \item \textbf{RC and RL circuits simulation on direct and alternating current}. Main goal of this line would be to learn how RC and RL circuits works, where the app user could build its own circuit. It could be interesting to show how differents current intensity, capacitors charge or magnetic flux at inductor evolves on time. The \textbf{simulink API} could be used for this purpose. At differential equations, we should consider the type of current we are using, so a new mathematical analysis for these circuits will be necessary. 
    
    \item \textbf{Physics simulations for teaching}. This line of research has the objective to extend this one, adding new simulations to the existing ones, such as the interaction between electric charges, or add new devices to RC and RL circuits like diodes or transistors.
\end{itemize}

\end{document}