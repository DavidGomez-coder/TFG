\documentclass[../main.tex]{subfiles}

\begin{document}

        \begin{longtable}{|| p{0.3\linewidth} | p{0.7\linewidth} ||}
            \hline
            \textbf{*} & El selector asterisco en CSS es utilizado para aplicar el estilo definido a cualquier elemento. Suele ser utilizado para realizar pequeñas pruebas, pero nunca es recomendado usarlo en la aplicación final. \\ \hline
            
            \textbf{A} & Este hace referencia a cualquier etiqueta predefinida de HTML o bien, a un elemento que nosotros mismos hayamos creado. \\ \hline
            
            \textbf{\#mi-id} & Aplicando el símbolo \textit{\#} delante de una cadena de caracteres, nos permite aplicar un formato a aquellos elementos que tengan como atributo definido un identificador (\textit{id=``mi-id''}). A diferencia de otros selectores, los identificadores deben únicos por documento HTML generado. \\ \hline
            
            \textbf{.mi-clase} & Si delante de una cadena alfanumérica colocamos punto, tenemos un selector por clase. El formato definido se aplica solamente a aquellos elementos que tienen un atributo \textit{class} asociado al valor \textit{mi-clase}. \\ \hline
            
            \textbf{A B} & Se trata de un selector de herencia. Los estilos se aplican a todos los elementos del tipo B que son descendientes directos de A. \\ \hline
            
            \textbf{A:visited / A:link} & La pseudoclase \textit{:visited} nos permite añadir un formato específico a aquellos elementos del tipo A a los que no se han clickado aún. Por el contrario, la pseudoclase A:link tiene el efecto contrario. \\ \hline
            
            \textbf{A + B} & Similar al selector \textit{A B}. En este caso, los estilos se aplican solamente al primer elemento del tipo B que es descendiente de A. \\ \hline
            
            \textbf{A > B} & Similar al selector \textit{A B}. En este caso los estilos se aplican solamente al primer hijo descendiente directo del tipo B de A. \\ \hline
            
            \textbf{A ~ B} & Con este selector basado en la herencia, el formato se aplicará a cualquier elemento del tipo B siempre que estén precedidos por uno de tipo A. \\ \hline
            
            \textbf{A[atr]} & Se trata de un selector por definición de atributos. Solamente se seleccionará aquellos elementos del tipo A que tengan definido un atributo con nombre \textit{atr}. \\ \hline
            
            \textbf{A[atr="val"]} & Para especificar el selector anterior, en este se tiene en cuenta el valor del atributo \textit{atr} a la hora de aplicar el estilo. \\ \hline
            
            \textbf{A:checked} & La pseusoclase solamente afecta a aquellos elementos que son del tipo selección quue han sido seleccionados (cómo los \textit{checkboxes} o \textit{radio buttons}). \\ \hline
            
            \textbf{A:before / A:after} & Estas pseudoclases tienen la particularidad de aplicar los estilos de un elemento A antes y después de la creación del mismo respectivamente. \\ \hline
            
            \textbf{A:hover} & Los estilos se aplican al pasar el cursor sobre el objeto A. \\ \hline
            
            \textbf{A:not(selector)} & Selector de negación. Afecta a todos los elementos del tipo A a excepción de aquellos que cumplen con la condición impuesta por el \textit{selector} que se encuentre entre paréntesis.. \\ \hline
            
            \caption{Estos son algunos de los selectores más comunes de  CSS. En \cite{selectoresCSS} se pueden ver con más detalle y con algunos ejemplos.}
            \label{selectores_css}
            
        \end{longtable}

        
        
    
        




   

\end{document}