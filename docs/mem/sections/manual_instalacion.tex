\documentclass[../main.tex]{subfiles}

\begin{document}

En primer lugar, debemos de asegurarnos de tener instalado NodeJS (\url{ https://nodejs.org/en/download/}). Una vez la instalación está completa, procedemos a descargar el código fuente del proyecto de git (este repositorio se encuentra en privado, por lo que se necesitará ser colaborador en el mismo u obtenerlo de otra forma).

\begin{lstlisting}[language=bash]
  $ git clone https://github.com/DavidGomez-coder/TFG.git
\end{lstlisting}

A continuación, debemos de abrir una consola de comandos en el directorio raíz del proyecto e instalar todas las dependencias utilizando el siguiente comando:

\begin{lstlisting}[language=bash]
  $ npm install
\end{lstlisting}

Una vez finalizado el proceso iniciamos la aplicación

\begin{lstlisting}[language=bash]
  $ npm start
\end{lstlisting}

, la cuál se abrirá en una nueva pestaña de nuestro navegador web por defecto en la dirección \url{http://localhost:3000/TFG}.




\end{document}