\documentclass[../main.tex]{subfiles}

\begin{document}




\subsection{Conclusiones}
Tal y como se comentó a lo largo del primer capítulo, el objetivo principal a cubrir en esta línea de trabajo era elaborar material docente para el alumnado de las asignaturas de física para reforzar la comprensión de los circuitos RC y RL de primer orden, a través de simulaciones informáticas.\\ 

Para el desarrollo de estas simulaciones se plantearon tres tipos de aplicaciones diferentes: nativas, híbridas y web, de las que finalmente la escogida fue la aplicación web, principalmente por su usabilidad en los diferentes dispositivos del mercado. Una vez hecho esto, entramos en el desarrollo de la propia aplicación, en la que cumplimos los siguientes objetivos en cada una de las etapas del mismo. \\

\begin{itemize}
    \item Durante la etapa de ańalisis, el diálogo con los profesores fue bastante importante, ya que de aquí se extrayeron todos los requisitos funcionales (tabla \ref{tab::requisitos}) que se integrarían en un futuro a la aplicación, además de una idea básica sobre el diseño de la misma.\\ 
    
    Además de esto, se llevó a cabo la tarea de investigar acerca de los fenómenos físicos propuestos y extraer las expresiones matemáticas que los modelan. 

    \item Tras realizar un análisis completo de la aplicación, se llevó a cabo un estudio en el que se barajó varios candidatos como posibles tecnologías a utilizar, como el tipo de aplicación y los lenguajes de programación. Al principio, el objetivo fue utilizar Java como lenguaje, aunque finalmente optamos por hacer uso de tecnologías más actuales del desarrollo web.\\ 
    
    En esta fase, tuve como objetivo aprender a dominar los diferentes \textit{frameworks} y lenguajes de programación que se utilizan en la implementación, aprendiendo prácticamente la mayoría desde cero.

    \item Una vez aprendidas las tecnologías, llegamos a la parte de implementación y a las pruebas de la aplicación. Durante esta etapa,  utilizamos las tecnologías de la etapa diseño y las expresiones obtenidas durante el análisis para desarrollar la aplicación.\\ 
    
    \item Por último, una vez la aplicación está completa, debemos de dar acceso a ella públicamente, realizando un despliegue del \textit{software} para su uso en un navegador web. Esta es la parte más sencilla, pues el entorno de desarrollo permite integrar una opción de despliegue bastante fácil de utilizar.
\end{itemize}

Con todo esto, tenemos una aplicación web completamente funcional, accesible desde casi cualquier dispositivo con un navegador web. La interfaz de usuario intuitiva, lo que permite al usuario navegar con facilidad a través de ella. Además, junto a las simulaciones, existen apartados teóricos que refuerzan el objetivo de la docencia, simulando completamente una clase teórica-práctica del funcionamiento de los circuitos RC y RL en corriente continua y estado transitorio.



\subsection{Líneas Futuras}
Como líneas futuras a este, se pueden proponer los siguientes.

\begin{itemize}
    \item \textbf{Portal para la asignatura de física. }Una de las líneas a proponer sería el desarrollo de un portal web en el que recoger toda la información sobre la asignatura; por ejemplo, las simulaciones realizadas, apartados teóricos y prácticos para el alumno, un sistema de \textit{login} para permitir el acceso solo a los alumnos de la asignatura, étc. 


    \item \textbf{Simulación de circuitos RC y RL en corriente continua y alterna}. Esta línea tendría como principal objetivo el aprendizaje de los circuitos RC y RL en general, en los que el alumno pueda construir su propio circuito. Podría ser interesante observar cómo evolucionan la intensidad de corriente en cada malla del circuito, la carga de todos los condensadores o el flujo magnético en las bobinas empleadas. Podría emplearse para ello la \textbf{API} de \textbf{simulink}, un entorno de programación de muy alto nivel de \textit{Matlab} usado normalmente para resolver ecuaciones diferenciales.
    
    \item \textbf{Simulaciones físicas para la docencia. }Esta línea de trabajo trataría de extender la solución propuesta en este. Podrían añadirse nuevas simulaciones simples a las ya existentes, como simular la interacción entre cargas eléctricas.
    

\end{itemize}




\end{document}