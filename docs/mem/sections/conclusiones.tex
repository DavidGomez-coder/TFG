\documentclass[../main.tex]{subfiles}

\begin{document}




\subsection{Conclusiones}
Tal y como se comentó a lo largo del primer capítulo, el objetivo principal a cubrir en esta línea de trabajo era elaborar material docente para el alumnado de las asignaturas de física para reforzar la comprensión de los circuitos RC y RL de primer orden, a través de simulaciones informáticas.





\begin{itemize}

    \item Se han analizado cual es la aplicación adecuada de entre las siguientes: nativas, híbridas y web. Con el objeto de extender su usabilidad a los diferentes dispositivos del mercado se ha escogido la aplicación web.

    \item Se ha desarrollado una primera etapa de análisis de las necesidades académicas de las que se han extraido los requisitos funcionales (tabla \ref{tab::requisitos}) y de diseño que se han integrado posteriormente a la aplicación informática; principalmente, la posibilidad de visualizar a tiempo real la evolución temporal de las magnitudes eléctricas y las expresiones matemáticas que justifican el comportamiento de las mismas.
    
    \item Tras realizar un análisis completo de la aplicación, se ha llevado a cabo un estudio de las diferentes tecnologías a utilizar, como el tipo de aplicación y los lenguajes de programación. Aunque inicialmente el objetivo era usar Java como lenguaje de programación, finalmente se ha optado por hacer uso de tecnologías más actuales del desarrollo web.\\
    
    En esta fase, se ha aprendido a dominar desde cero los diferentes \textit{framewoks} y lenguajes empleados en la implementación. Entre estas tecnologías, tenemos \textit{ReactJS}, un \textit{framework frontend} orientado a la programación de vistas basado en la construcción de componentes. Puesto que esta librería se encuentra escrita en \textit{JavaScript}, este es el lenguaje utilizado internamente en el proyecto, junto \textit{HTML} y \textit{CSS}, esenciales en la programación web.\\

    Se ha usado la versión web de git: \textit{GitHub}, para llevar así un control sobre las actualizaciones del proyecto. Este también permite configurarlo para poder realizar despliegues del código desarrollado, de tal forma que la aplicación final sea accesible desde cualquier navegador, facilitando el acceso a la web a todo el mundo.
    
    \item Se ha realizado la implementación de la aplicación, dónde lo más destacable es el algoritmo utilizado en la generación de los datos, en el que aprovechamos el ciclo de vida de un componente en \textit{react} para generar, tratar y representar los resultados correspondientes a cada una de las magnitudes físicas de la simulación. 
    
    \item Se ha realizado la comprobación de que el funcionamiento de la aplicación es acorde a los requisitos establecidos. Ya que se trata de una aplicación completamente visual y lo que nos interesa es que las gráficas generadas sean las correctas, las pruebas realizadas no son más que una comparación entre los resultados generados por la aplicación y los obtenidos por un \textit{script} de \textit{python}. 
    
    
    \item Una vez la aplicación está completa, debemos de dar acceso a ella públicamente, realizando un despliegue del \textit{software} para su uso en un navegador web.
    
    \item Se ha desarrollado una aplicación web completamente funcional acorde a las necesidades especificadas, accesible desde casi cualquier dispositivo con un navegador web. La interfaz de usuario intuitiva, lo que permite al usuario navegar con facilidad a través de ella. Además, junto a las simulaciones, existen apartados teóricos que refuerzan el objetivo de la docencia, simulando completamente una clase teórica-práctica del funcionamiento de los circuitos RC y RL en corriente continua y estado transitorio.
    
\end{itemize}






\subsection{Líneas Futuras}
Como líneas futuras a este, se pueden proponer los siguientes.

\begin{itemize}
    \item \textbf{Portal para la asignatura de física. }Una de las líneas a proponer sería el desarrollo de un portal web en el que recoger toda la información sobre la asignatura; por ejemplo, las simulaciones realizadas, apartados teóricos y prácticos para el alumno, un sistema de \textit{login} para permitir el acceso solo a los alumnos de la asignatura, étc. 


    \item \textbf{Simulación de circuitos RC y RL en corriente continua y alterna}. Esta línea tendría como principal objetivo el aprendizaje de los circuitos RC y RL en general, en los que el alumno pueda construir su propio circuito. Podría ser interesante observar cómo evolucionan la intensidad de corriente en cada malla del circuito, la carga de todos los condensadores o el flujo magnético en las bobinas empleadas. Podría emplearse para ello la \textbf{API} de \textbf{simulink}, un entorno de programación de muy alto nivel de \textit{Matlab} usado normalmente para resolver ecuaciones diferenciales. En este caso, las ecuaciones a implementar tendríamos que tener en cuenta el tipo de corriente utilizada, por lo que habría que relizar un análisis de estos circuitos para corriente alterna.
    
    \item \textbf{Simulaciones físicas para la docencia. }Esta línea de trabajo trataría de extender la solución propuesta en este. Podrían añadirse nuevas simulaciones simples a las ya existentes, como simular la interacción entre cargas eléctricas o añadir nuevos dispositivos a los circuitos existentes, como diodos o transistores.
    

\end{itemize}




\end{document}