\documentclass[../main.tex]{subfiles}

\begin{document}
%% begin abstract format
\makeatletter
\renewenvironment{abstract}{%
    \if@twocolumn
      \section*{Resumen \\}%
    \else %% <- here I've removed \small
    \begin{flushright}
        {\filleft\Huge\bfseries\fontsize{48pt}{12}\selectfont Resumen\vspace{\z@}}%  %% <- here I've added the format
        \end{flushright}
      \quotation
    \fi}
    {\if@twocolumn\else\endquotation\fi}
\makeatother
%% end abstract format
%% begin abstract format
\makeatletter
\renewenvironment{abstract}{%
    \if@twocolumn
      \section*{Resumen \\}%
    \else %% <- here I've removed \small
    \begin{flushright}
        {\filleft\Huge\bfseries\fontsize{48pt}{12}\selectfont Resumen\vspace{\z@}}%  %% <- here I've added the format
        \end{flushright}
      \quotation
    \fi}
    {\if@twocolumn\else\endquotation\fi}
\makeatother
%% end abstract format
\begin{abstract}
La mayor parte de los alumnos que cursan alguna asignatura de física en ramas técnicas como en ingeniería o arquitectura, suelen encontrar cierta complejidad al intentar comprender los temas estudiados. Esto no se debe a la dificultad matemática de los problemas que se plantean, sino más bien en interpretar la evolución de los fenómenos físicos que se estudian. 

Así que para tratar de solucionar este problema y facilitar su estudio, se plantea el desarrollo de una aplicación web que sea capaz de mostrar la evolución de diferentes magnitudes físicas de los circuitos RC y RL en corriente continua y estado transitorio acompañada de una explicación teórica sobre las mismas.\\


\bfseries{\large{Palabras clave:}} simulación, circuito RC, circuito RL, corriente continua, estado transitorio

\end{abstract}
\end{document}


