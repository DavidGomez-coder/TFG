\documentclass[../main.tex]{subfiles}

\begin{document}
%% begin abstract format
\makeatletter
\renewenvironment{abstract}{%
    \if@twocolumn
      \section*{Resumen \\}%
    \else %% <- here I've removed \small
    \begin{flushright}
        {\filleft\Huge\bfseries\fontsize{48pt}{12}\selectfont Resumen\vspace{\z@}}%  %% <- here I've added the format
        \end{flushright}
      \quotation
    \fi}
    {\if@twocolumn\else\endquotation\fi}
\makeatother
%% end abstract format
%% begin abstract format
\makeatletter
\renewenvironment{abstract}{%
    \if@twocolumn
      \section*{Resumen \\}%
    \else %% <- here I've removed \small
    \begin{flushright}
        {\filleft\Huge\bfseries\fontsize{48pt}{12}\selectfont Resumen\vspace{\z@}}%  %% <- here I've added the format
        \end{flushright}
      \quotation
    \fi}
    {\if@twocolumn\else\endquotation\fi}
\makeatother
%% end abstract format
\begin{abstract}
La mayor parte de los alumnos que cursan alguna asignatura de física en ramas técnicas como en ingeniería o arquitectura, suelen encontrar cierta complejidad al intentar comprender los temas estudiados. Esto no se debe a la dificultad matemática de los problemas que se plantean, sino más bien en interpretar la evolución de los fenómenos físicos que se estudian.\\ 

Así que para tratar de minimizar este problema, tomando como sistemas de estudio los circuitos RC y RL de uso común en dispositivos eléctricos y electrónicos, se plantea el desarrollo de una aplicación web que sea capaz de mostrar claramente las expresiones teóricas que explican su comportamiento y analizar la evolución temporal de las diferentes magnitudes físicas de estos sistemas cuando trabajan en corriente continua y estado transitorio.\\

Abordaremos este problema utilizando las últimas tecnologías del desarrollo web; utilizando lenguajes como \textit{JavaScript} y entornos de desarrollo como \textit{NodeJS}. Además, para comprobar que los resulados de la aplicación son correctos se implementará un pequeño programa en \textit{python}, el que generará una imagen con los resultados que debemos de esperar de la simulación. \\

Finalmente, una vez establecidas las conclusiones acerca del desarrollo y análisis de esta aplicación, se propondrá su posible adaptación a otros sistemas físicos similares a los circuitos estudiados.\\

\bfseries{\large{Palabras clave:}} simulación, circuito RC, circuito RL, corriente continua, estado transitorio, ReactJS, aplicación web 

\end{abstract}
\end{document}


